\documentclass[12pt]{article}
\input{../../app_style.tex}

%%%%%%%%%%%
% App Ex number    %
%%%%%%%%%%%

% DON'T FORGET TO UPDATE

\newcommand{\appno}[1]
{3.2}

%%%%%%%%%%%%%%
% Turn on/off solutions       %
%%%%%%%%%%%%%%

% Off
%\newcommand{\soln}[2]{$\:$\\ \vspace{#1}}{}

%%% On
\newcommand{\soln}[2]{\textit{\textcolor{custom_red}{#2}}}{}

%%%%%%%%%%%%%%%%
% Document
%%%%%%%%%%%%%%%%

\begin{document}
%\fontspec[Ligatures=TeX]{Helvetica Neue Light}
Sergio Garcia Rios\hfill GOVT 3990 Puzzle Solving with Data \\
Cornell University\hfill \\

\ttl{Application exercise \appno{}: \\
Grade inflation}

\inst{$\:$ \\
Team names: \rule{10cm}{0.5pt} \\

Write your responses in the spaces provided below. WRITE LEGIBLY and SHOW ALL WORK! 
Only one submission per team is required. }

%%%%%%%%%%%%%%%%%%%%%%%%%%%%%%%%%%%%

In 2001 the average GPA of students at Cornell University was 3.37. Last semester 63 GOVT 3990 students responded to the question on GPA on the class survey. The mean was 3.58, and the 
standard deviation 0.53. A histogram of the data is shown below.

\begin{center}
\includegraphics[width=0.5\textwidth]{survey/hist_gpa}
\end{center}

Assuming that this sample is random and representative of all Cornell students (bit of a leap 
of faith? you can discuss that when checking the conditions), do these data provide 
convincing evidence that the average GPA of Cornell students has \textbf{\underline{changed}} 
over the last decade and a half?

Make sure to check conditions, note any assumptions you make, and show all your work.

\soln{4cm}{
Conditions:
\begin{itemize}
\item The sample is less than 10\% of the population of all Cornell students, however it is not random.
It may not be reasonable to assume that the GPAs of sampled Cornell students are independent
of each other.
\item The distribution is not extremely skewed, and the sample size is sufficiently large, to yield a nearly
normal sampling distribution of the mean.
\end{itemize}
\[ Z = \frac{3.58 - 3.37}{0.53 / \sqrt{63}} = 3.15 \rightarrow p-value = 0.0016 \]
}

%%%%%%%%%%%%%%%%%%%%%%%%%%%%%%%%%%%%

\end{document}