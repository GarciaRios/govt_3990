\documentclass[12pt]{article}
%%%%%%%%%%%%%%%%
% Packages
%%%%%%%%%%%%%%%%

\usepackage[top=1.5cm,bottom=1.5cm,left=1.5cm,right= 1.5cm]{geometry}
\usepackage[parfill]{parskip}
\usepackage{graphicx, fontspec, xcolor,multicol, enumitem, setspace, amsmath, changepage}
\DeclareGraphicsRule{.tif}{png}{.png}{`convert #1 `dirname #1`/`basename #1 .tif`.png}

%%%%%%%%%%%%%%%%
% User defined colors
%%%%%%%%%%%%%%%%

% Pantone 2015 Fall colors
% http://iwork3.us/2015/02/18/pantone-2015-fall-fashion-report/
% update each semester or year

\xdefinecolor{custom_blue}{rgb}{0, 0.32, 0.48} % FROM SPRING 2016 COLOR PREVIEW
\xdefinecolor{custom_darkBlue}{rgb}{0.20, 0.20, 0.39} % Reflecting Pond  
\xdefinecolor{custom_orange}{rgb}{0.96, 0.57, 0.42} % Cadmium Orange
\xdefinecolor{custom_green}{rgb}{0, 0.47, 0.52} % Biscay Bay
\xdefinecolor{custom_red}{rgb}{0.58, 0.32, 0.32} % Marsala

\xdefinecolor{custom_lightGray}{rgb}{0.78, 0.80, 0.80} % Glacier Gray
\xdefinecolor{custom_darkGray}{rgb}{0.35, 0.39, 0.43} % Stormy Weather

%%%%%%%%%%%%%%%%
% Color text commands
%%%%%%%%%%%%%%%%

%orange
\newcommand{\orange}[1]{\textit{\textcolor{custom_orange}{#1}}}

% yellow
\newcommand{\yellow}[1]{\textit{\textcolor{yellow}{#1}}}

% blue
\newcommand{\blue}[1]{\textit{\textcolor{blue}{#1}}}

% green
\newcommand{\green}[1]{\textit{\textcolor{custom_green}{#1}}}

% red
\newcommand{\red}[1]{\textit{\textcolor{custom_red}{#1}}}

%%%%%%%%%%%%%%%%
% Coloring titles, links, etc.
%%%%%%%%%%%%%%%%

\usepackage{titlesec}
\titleformat{\section}
{\color{custom_blue}\normalfont\Large\bfseries}
{\color{custom_blue}\thesection}{1em}{}
\titleformat{\subsection}
{\color{custom_blue}\normalfont}
{\color{custom_blue}\thesubsection}{1em}{}

\newcommand{\ttl}[1]{ \textsc{{\LARGE \textbf{{\color{custom_blue} #1} } }}}

\newcommand{\tl}[1]{ \textsc{{\large \textbf{{\color{custom_blue} #1} } }}}

\usepackage[colorlinks=false,pdfborder={0 0 0},urlcolor= custom_orange,colorlinks=true,linkcolor= custom_orange, citecolor= custom_orange,backref=true]{hyperref}

%%%%%%%%%%%%%%%%
% Instructions box
%%%%%%%%%%%%%%%%

\newcommand{\inst}[1]{
\colorbox{custom_blue!20!white!50}{\parbox{\textwidth}{
	\vskip10pt
	\leftskip10pt \rightskip10pt
	#1
	\vskip10pt
}}
\vskip10pt
}

%%%%%%%%%%%
% App Ex number    %
%%%%%%%%%%%

% DON'T FORGET TO UPDATE

\newcommand{\appno}[1]
{3.1}

%%%%%%%%%%%%%%
% Turn on/off solutions       %
%%%%%%%%%%%%%%

% Off
%\newcommand{\soln}[2]{$\:$\\ \vspace{#1}}{}

%%%% On
\newcommand{\soln}[2]{\textit{\textcolor{custom_red}{#2}}}{}

%%%%%%%%%%%%%%%%
% Document
%%%%%%%%%%%%%%%%

\begin{document}
%\fontspec[Ligatures=TeX]{Helvetica Neue Light}

Sergio I Garcia-Rios \hfill GOVT 3990 Puzzle Solving with Data \\
Cornell University - Department of Government \hfill \\

\ttl{Application exercise \appno{}: \\
Relaxing after work}

\inst{$\:$ \\
Team name: \rule{10cm}{0.5pt} \\
$\:$ \\
Lab section: $\qquad$ 8:30 $\qquad$ 10:05 $\qquad$ 11:45 $\qquad$ 1:25 $\qquad$ 3:05 $\qquad$ 4:40 \\
$\:$ \\
Write your responses in the spaces provided below. WRITE LEGIBLY and SHOW ALL WORK! 
Only one submission per team is required.}

The General Social Survey (GSS) is a sociological survey used to collect data on demographic characteristics 
and attitudes of residents of the United States. In 2010, the survey collected responses from 1,154 US residents. 
The survey is conducted face-to-face with an in-person interview of a randomly-selected sample of adults. One 
of the questions on the survey is ``After an average work day, about how many hours do you have to relax or 
pursue activities that you enjoy?". The average time spent relaxing was 3.68 hours, with a standard deviation 
of 2.6 hours.

\begin{enumerate}

\item Is the distribution of number of hours spent relaxing after work for Americans nearly normal? 
How can you tell?

\soln{2cm}{Right skewed since the mean is too close to the natural boundary at 0 for how large the standard
deviation is.}

\item If your answer to the previous question is no, can we still use CLT based methods to estimate 
the true average number of hours spent relaxing after work for Americans using these data? Why, 
or why not?

\soln{4cm}{Yes, because the following conditions check out:
\begin{itemize}
\item Independence: Since the sample is random and less than 10\% of the population, we can assume that
the sampled individuals are independent of each other with respect to how much they relax after work.
\item Sample size / skew: While the distribution of the sample (and hence the population) is right skewed,
we have a large enough sample for the sampling distribution of the sample mean to be nearly normal.
\end{itemize}
}

\pagebreak

\item Construct a 95\% confidence interval for the true average number of hours spent relaxing after 
work for Americans.

\soln{3.5cm}{
\[ 
3.68 \pm 1.96 \times \frac{2.6}{\sqrt{1,154}} = 3.68 \pm 0.15 = (3.53,~3.83)
\]
}

\item Interpret this interval in context of the data.

\soln{3cm}{We are 95\% confident that Americans relax on average 3.53 to 3.83 hours after work.}

\item What does ``95\% confident" mean in your interpretation?

\soln{3cm}{95\% of random samples of size 1,154 from this population will yield confidence intervals
that contain the true population mean of number of hours Americans spend relaxing after work.}

\item Would you expect a 90\% confidence interval to be wider or narrower than the 95\% confidence 
interval you reported in the previous question? Why?

\soln{2cm}{We would expect the 90\% interval to be narrower. Since we don't have to be as accurate,
we can be more precise. Also the $z^\star$ for the 90\% confidence interval is smaller, 1.65 vs. 1.96.}

\item Calculate the sample size necessary to obtain a 90\% confidence interval with a margin of error of
0.06 hours.

\soln{4cm}{At least 5,113 people.
\[ 
0.06 \le 1.65 \times \frac{2.6}{\sqrt{n}} \rightarrow n \ge \left( \frac{1.65 \times 2.6}{0.06} \right)^2 \rightarrow n \ge 5112.25
\]
}

\end{enumerate}

%%%%%%%%%%%%%%%%%%%%%%%%%%%%%%%%%%%%

\end{document}