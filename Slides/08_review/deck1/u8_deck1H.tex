% -*- TeX-engine: xetex; eval: (auto-fill-mode 0); eval: (visual-line-mode 1); -*-
% Compile with XeLaTeX

%%%%%%%%%%%%%%%%%%%%%%%
% Option 1: Slides: (comment for handouts)   %
%%%%%%%%%%%%%%%%%%%%%%%

%\documentclass[slidestop,compress,mathserif,12pt,t,professionalfonts,xcolor=table]{beamer}
%
%% solution stuff
%\newcommand{\solnMult}[1]{
%\only<1>{#1}
%\only<2->{\red{\textbf{#1}}}
%}
%\newcommand{\soln}[1]{\textit{#1}}

%%%%%%%%%%%%%%%%%%%%%%%%%%%%%%%
% Option 2: Handouts, without solutions (post before class)    %
%%%%%%%%%%%%%%%%%%%%%%%%%%%%%%%

 \documentclass[11pt,containsverbatim,handout,xcolor=xelatex,dvipsnames,table]{beamer}

 % handout layout
 \usepackage{pgfpages}
 \pgfpagesuselayout{4 on 1}[letterpaper,landscape,border shrink=5mm]

% % solution stuff
 \newcommand{\solnMult}[1]{#1}
 \newcommand{\soln}[1]{}

%%%%%%%%%%%%%%%%%%%%%%%%%%%%%%%%%%%%
% Option 3: Handouts, with solutions (may post after class if need be)    %
%%%%%%%%%%%%%%%%%%%%%%%%%%%%%%%%%%%%

% \documentclass[11pt,containsverbatim,handout,xcolor=xelatex,dvipsnames,table]{beamer}

% % handout layout
% \usepackage{pgfpages}
% \pgfpagesuselayout{4 on 1}[letterpaper,landscape,border shrink=5mm]

% % solution stuff
% \newcommand{\solnMult}[1]{\red{\textbf{#1}}}
% \newcommand{\soln}[1]{\textit{#1}}

%%%%%%%%%%
% Load style file, defaults  %
%%%%%%%%%%

%%%%%%%%%%%%%%%%
% Themes
%%%%%%%%%%%%%%%%

% See http://deic.uab.es/~iblanes/beamer_gallery/ for mor options

% Style theme
\usetheme{metropolis}

% Color theme
%\usecolortheme{seahorse}

% Helvetica Neue Light for most text
%\usepackage{fontspec}
%\setsansfont{Helvetica Neue Light}

%%%%%%%%%%%%%%%%
% Packages
%%%%%%%%%%%%%%%%

\usepackage{geometry}
\usepackage{graphicx}
\usepackage{amssymb}
\usepackage{epstopdf}
\usepackage{amsmath}  	% this permits text in eqnarray among other benefits
\usepackage{url}		% produces hyperlinks
\usepackage[english]{babel}
\usepackage{colortbl}	% allows for color usage in tables
\usepackage{multirow}	% allows for rows that span multiple rows in tables
\usepackage{color}		% this package has a variety of color options
\usepackage{pgf}
\usepackage{calc}
\usepackage{ulem}
\usepackage{multicol}
\usepackage{textcomp}
\usepackage{listings}
\usepackage{changepage}
\usepackage{tikz}
\usetikzlibrary{trees}		% for probability trees
\usepackage{fancyvrb}	% for colored code chunks
\usepackage{nameref}

%%%%%%%%%%%%%%%%
% Remove navigation symbols
%%%%%%%%%%%%%%%%

\beamertemplatenavigationsymbolsempty
\hypersetup{pdfpagemode=UseNone} % don't show bookmarks on initial view

%%%%%%%%%%%%%%%%
% User defined colors
%%%%%%%%%%%%%%%%

% Pantone 2016 Spring colors
% https://atelierbram.github.io/c-tiles16/colorscheming/pantone-spring-2016-colortable.html
% update each semester or year

\xdefinecolor{custom_blue}{rgb}{0.01, 0.31, 0.52} % Snorkel Blue
\xdefinecolor{custom_darkBlue}{rgb}{0.20, 0.20, 0.39} % Reflecting Pond  
\xdefinecolor{custom_orange}{rgb}{0.96, 0.57, 0.42} % Cadmium Orange
\xdefinecolor{custom_green}{rgb}{0, 0.47, 0.52} % Biscay Bay
\xdefinecolor{custom_red}{rgb}{0.58, 0.32, 0.32} % Marsala

\xdefinecolor{custom_lightGray}{rgb}{0.78, 0.80, 0.80} % Glacier Gray
\xdefinecolor{custom_darkGray}{rgb}{0.35, 0.39, 0.43} % Stormy Weather

%%%%%%%%%%%%%%%%
% Template colors
%%%%%%%%%%%%%%%%

%\setbeamercolor*{palette primary}{fg=white,bg= custom_blue}
%\setbeamercolor*{palette secondary}{fg=black,bg= custom_blue!80!black}
%\setbeamercolor*{palette tertiary}{fg=white,bg= custom_blue!80!black!80}
%\setbeamercolor*{palette quaternary}{fg=white,bg= custom_blue}
%
%\setbeamercolor{structure}{fg= custom_blue}
%\setbeamercolor{frametitle}{bg= custom_blue!90}
%\setbeamertemplate{blocks}[shadow=false]
%\setbeamersize{text margin left=2em,text margin right=2em}

%%%%%%%%%%%%%%%%
% Styling fonts, bullets, etc.
%%%%%%%%%%%%%%%%
%
%% title slide
%\setbeamerfont{title}{size=\large,series=\bfseries}
%\setbeamerfont{subtitle}{size=\large,series=\mdseries}
%%\setbeamerfont{institute}{size=\large,series=\mdseries}
%
% color of alerted text
\setbeamercolor{alerted text}{fg=custom_orange}

% styling of itemize bullets
\setbeamercolor{item}{fg=custom_blue}
\setbeamertemplate{itemize item}{{{\small$\blacktriangleright$}}}
\setbeamercolor{subitem}{fg=custom_blue}
\setbeamertemplate{itemize subitem}{{\textendash}}
\setbeamerfont{itemize/enumerate subbody}{size=\footnotesize}
\setbeamerfont{itemize/enumerate subitem}{size=\footnotesize}

% styling of enumerate bullets
\setbeamertemplate{enumerate item}{\insertenumlabel.}

%\setbeamerfont{enumerate item}{family={\fontspec{Helvetica Neue}}}
%\setbeamerfont{enumerate subitem}{family={\fontspec{Helvetica Neue}}}
%\setbeamerfont{enumerate subsubitem}{family={\fontspec{Helvetica Neue}}}

% make frame titles small to make room in the slide
\setbeamerfont{frametitle}{size=\small} 




%% set Helvetica Neue font for frame and section titles
%\setbeamerfont{frametitle}{family={\fontspec{Helvetica Neue}}}
%\setbeamerfont{sectiontitle}{family={\fontspec{Helvetica Neue}}}
%\setbeamerfont{section in toc}{family={\fontspec{Helvetica Neue}}}
%\setbeamerfont{subsection in toc}{family={\fontspec{Helvetica Neue}}, size=\small}
%\setbeamerfont{footline}{family={\fontspec{Helvetica Neue}}}
%\setbeamerfont{subsection in toc}{family={\fontspec{Helvetica Neue}}}
%\setbeamerfont{block title}{family={\fontspec{Helvetica Neue}}}
%
%%%%%%%%%%%%%%%%%
%% New fonts accessed by fontspec package
%%%%%%%%%%%%%%%%%
%
%% Monaco font for code
%\newfontfamily{\monaco}{Monaco}

%%%%%%%%%%%%%%%%
% Color text commands
%%%%%%%%%%%%%%%%

%orange
\newcommand{\orange}[1]{\textit{\textcolor{custom_orange}{#1}}}

% yellow
\newcommand{\yellow}[1]{\textit{\textcolor{yellow}{#1}}}

% blue
\newcommand{\blue}[1]{\textit{\textcolor{blue}{#1}}}

% green
\newcommand{\green}[1]{\textit{\textcolor{custom_green}{#1}}}

% red
\newcommand{\red}[1]{\textit{\textcolor{custom_red}{#1}}}

% dark gray
\newcommand{\darkgray}[1]{\textit{\textcolor{custom_darkGray}{#1}}}

% light gray
\newcommand{\lightgray}[1]{\textit{\textcolor{custom_lightGray}{#1}}}

% pink
\newcommand{\pink}[1]{\textit{\textcolor{pink}{#1}}}


%%%%%%%%%%%%%%%%
% Custom commands
%%%%%%%%%%%%%%%%

% empty box for probability tree frame
\newcommand{\emptybox}[2]{
	\fbox{ \begin{minipage}{#1} \hfill\vspace{#2} \end{minipage} }
}

% cancel
\newcommand{\cancel}[1]{%
    \tikz[baseline=(tocancel.base)]{
        \node[inner sep=0pt,outer sep=0pt] (tocancel) {#1};
        \draw[red, line width=0.5mm] (tocancel.south west) -- (tocancel.north east);
    }%
}

% degree
\newcommand{\degree}{\ensuremath{^\circ}}

% cite
\newcommand{\ct}[1]{
\vfill
{\tiny #1}}

% Note
\newcommand{\Note}[1]{
\rule{2.5cm}{0.25pt} \\ \textit{\footnotesize{\textcolor{custom_red}{Note:} \textcolor{custom_darkGray}{#1}}}}

% Remember
\newcommand{\Remember}[1]{\textit{\scriptsize{\textcolor{custom_red}{Remember:} #1}}}

% links: webURL, webLink
\newcommand{\webURL}[1]{\urlstyle{same}{\textit{\textcolor{custom_blue}{\url{#1}}}}}
\newcommand{\webLink}[2]{\href{#1}{\textcolor{custom_blue}{{#2}}}}

% mail
\newcommand{\mail}[1]{\href{mailto:#1}{\textit{\textcolor{custom_blue}{#1}}}}

% highlighting: hl, hlGr, mathhl
\newcommand{\hl}[1]{\textit{\textcolor{custom_blue}{#1}}}
\newcommand{\hlGr}[1]{\textit{\textcolor{custom_green}{#1}}}
\newcommand{\mathhl}[1]{\textcolor{custom_blue}{\ensuremath{#1}}}

% example
\newcommand{\ex}[1]{\textcolor{blue}{{{\small (#1)}}}}

% twocol: two columns
\newenvironment{twocol}[4]{
\begin{columns}[c]
\column{#1\textwidth}
#3
\column{#2\textwidth}
#4
\end{columns}
}

% threecol: three columns
\newenvironment{threecol}[6]{
\begin{columns}[c]
\column{#1\textwidth}
#4
\column{#2\textwidth}
#5
\column{#3\textwidth}
#6
\end{columns}
}

% slot (for probability calculations)
\newenvironment{slot}[2]{
\begin{array}{c} 
\underline{#1} \\ 
#2
\end{array}
}

% pr: left and right parentheses
\newcommand{\pr}[1]{
\left( #1 \right)
}

%%%%%%%%%%%%%%%%
% Custom blocks
%%%%%%%%%%%%%%%%

% activity: less commonly used
\newcommand{\activity}[2]{
\setbeamertemplate{itemize item}{{{\small\textcolor{custom_orange}{$\blacktriangleright$}}}}
\setbeamercolor{block title}{fg=white, bg=custom_orange}
\setbeamerfont{block title}{size=\small}
\setbeamercolor{block body}{fg=black, bg=custom_orange!20!white!80}
\setbeamerfont{block body}{size=\small}
\begin{block}{Activity: #1}
\setlength\abovedisplayskip{0pt}
#2
\end{block}
}

% app: application exercise
\newcommand{\app}[2]{
\setbeamercolor{block title}{fg=white,bg=custom_green}
\setbeamercolor{block body}{fg=black,bg=custom_green!20!white!80}
\begin{block}{{\small Application exercise: #1}}
#2
\end{block}
}

% disc: discussion question
\newcommand{\disc}[1]{
\vspace*{-2ex}
\setbeamercolor{block body}{bg=custom_blue!25!white!80, fg=custom_blue!55!black!95}
\begin{block}{\vspace*{-3ex}}
#1
\end{block}
\vspace*{-1ex}
}

% clicker: clicker question
\newcommand{\clicker}[1]{
\setbeamercolor{block title}{bg=custom_blue!80!white!50,fg=custom_blue!30!black!90}
\setbeamercolor{block body}{bg=custom_blue!20!white!80,fg=custom_blue!30!black!90}
\begin{block}{\vspace*{-0.2ex}{\footnotesize Your turn}\vspace*{-0.2ex}}
#1
\end{block}
}

% formula
\newcommand{\formula}[2]{
\setbeamercolor{block title}{bg=custom_blue!40!white!60,fg=custom_blue!55!black!95}
\begin{block}{{\small#1}}
#2
\end{block}
}

% code
\newcommand{\Rcode}[1]{
{\monaco {\footnotesize \textcolor{custom_darkBlue}{#1}}}
}

% output
\newcommand{\Rout}[1]{
{\monaco {\footnotesize \textcolor{custom_darkGray}{#1}}}
}

%%%%%%%%%%%%%%%%
% Change margin
%%%%%%%%%%%%%%%%

\newenvironment{changemargin}[2]{%
\begin{list}{}{%
\setlength{\topsep}{0pt}%
\setlength{\leftmargin}{#1}%
\setlength{\rightmargin}{#2}%
\setlength{\listparindent}{\parindent}%
\setlength{\itemindent}{\parindent}%
\setlength{\parsep}{\parskip}%
}%
\item}{\end{list}}

%%%%%%%%%%%%%%%%
% Footnote
%%%%%%%%%%%%%%%%

\long\def\symbolfootnote[#1]#2{\begingroup%
\def\thefootnote{\fnsymbol{footnote}}\footnote[#1]{#2}\endgroup}

%%%%%%%%%%%%%%%%
% Graphics
%%%%%%%%%%%%%%%%

\DeclareGraphicsRule{.tif}{png}{.png}{`convert #1 `dirname #1`/`basename #1 .tif`.png}

%%%%%%%%%%%%%%%%
% Slide number
%%%%%%%%%%%%%%%%

\setbeamertemplate{footline}{%
    \raisebox{5pt}{\makebox[\paperwidth]{\hfill\makebox[20pt]{\color{gray}
          \scriptsize\insertframenumber}}}\hspace*{5pt}}

          
%%%%%%%%%%%%%%%%
% Remove page numbers
%%%%%%%%%%%%%%%%

\newcommand{\removepagenumbers}{% 
  \setbeamertemplate{footline}{}
}

%%%%%%%%%%%%%%%%
% TOC slides
%%%%%%%%%%%%%%%%

\setbeamertemplate{section in toc}{\inserttocsectionnumber.~\inserttocsection}
\setbeamertemplate{subsection in toc}{$\qquad$\inserttocsubsectionnumber.~\inserttocsubsection \\}

\AtBeginSection[] 
{ 
  \addtocounter{framenumber}{-1} 
  % 
  {\removepagenumbers 
  {\small
    \begin{frame}<beamer> 
    \frametitle{Outline} 
    \tableofcontents[currentsection] 
  \end{frame} 
  } 
  }
} 

\AtBeginSubsection[] 
{ 
  \addtocounter{framenumber}{-1} 
  % 
  {\removepagenumbers 
  {\small
    \begin{frame}<beamer> 
    \frametitle{Outline} 
    \tableofcontents[currentsection,currentsubsection] 
  \end{frame} 
  } 
  }
}
% Course Name
\newcommand{\CourseName}{GOVT 3990 - Spring 2017}
\newcommand{\InstituteName}{Cornell University}

% Personal Info
\newcommand{\FirstName}{Sergio}
\newcommand{\LastName}{Garcia-Rios}

% Electronic Info
\newcommand{\PersonalSite}{https://garciarios.github.io}
\newcommand{\CourseSite}{http://garciarios.github.io/govt_3990/}
\newcommand{\Email}{garcia.rios@cornell.edu}

% Exam Dates
\newcommand{\ExamADate}{Feb 24, Wed}
\newcommand{\ExamBDate}{Mar 30, Wed}
\newcommand{\FinalDate}{May 5, Thu - 7-10pm}
% ALT ALT
% % Course Name
\newcommand{\CourseName}{Sta 101 - Spring 2016}
\newcommand{\InstituteName}{Duke University, Department of Statistical Science}

% Personal Info
\newcommand{\FirstName}{Anthea}
\newcommand{\LastName}{Monod}

% Electronic Info
\newcommand{\PersonalSite}{https://stat.duke.edu/people/anthea-monod.html}
\newcommand{\CourseSite}{https://stat.duke.edu/courses/Spring16/sta101.002}
\newcommand{\Email}{anthea@stat.duke.edu}

% Exam Dates
\newcommand{\ExamADate}{Feb 25, Thu}
\newcommand{\ExamBDate}{Mar 31, Thu}
\newcommand{\FinalDate}{???}

%%%%%%%%%%%
% Cover slide info    %
%%%%%%%%%%%

\title{Unit 8: Final Review}
\subtitle{1. Bayesian vs. frequentist inference}
\author{\CourseName}
\date{}
\institute{\InstituteName}


%%%%%%%%%%%%%%%%%%%%%%%%%
% Begin document and set Helvetica Neue font   %
%%%%%%%%%%%%%%%%%%%%%%%%%

\begin{document}
\fontspec[Ligatures=TeX]{Helvetica Neue Light}

%%%%%%%%%%%%%%%%%%%%%%%%%%%%%%%%%%%

% Title Page

\begin{frame}[plain]

\titlepage

\vfill

{\scriptsize \webLink{\PersonalSite}{Dr. \LastName{}} \hfill Slides posted at  \webURL{\CourseSite}}

\addtocounter{framenumber}{-1} 

\end{frame}

%%%%%%%%%%%%%%%%%%%%%%%%%%%%%%%%%%%%

\section{Housekeeping}

%%%%%%%%%%%%%%%%%%%%%%%%%%%%%%%%%%%%

\begin{frame}
\frametitle{Announcements}

\begin{itemize}

\item Project questions?

\end{itemize}

\end{frame}

%%%%%%%%%%%%%%%%%%%%%%%%%%%%%%%%%%%%

\section{Bayesian vs. Frequentist Inference}

%%%%%%%%%%%%%%%%%%%%%%%%%%%%%%%%%%%%

\begin{frame}
\frametitle{M\&Ms}

\begin{itemize}

\item We have a population of M\&Ms. The percentage of yellow M\&Ms is either 10\% or 20\%.

\item You have been hired as a statistical consultant to decide whether the true percentage of yellow M\&Ms is 10\%. You are being asked to make a decision, and there are associated payoff/losses that you should consider.

\end{itemize}

\end{frame}

%%%%%%%%%%%%%%%%%%%%%%%%%%%%%%%%%%%%

\begin{frame}
\frametitle{Decision table}

\begin{center}
\renewcommand\arraystretch{1.5}
\begin{tabular}{l | p{3cm} | p{3cm}}
				&   \multicolumn{2}{c}{True state of the population} \\
\hline
Decision			& \% yellow = 10\%		& \% yellow = 20\% \\
\hline
\% yellow = 10\%	& \green{Your boss gives you a bonus, and I bring you candy on Monday}	& \red{You lose your job, and no candy for you} \\
\hline
\%yellow = 20\%	& \red{You lose your job, and no candy for you} 	& \green{Your boss gives you a bonus, and I bring you candy on Monday} \\
\end{tabular}
\end{center}

\end{frame}

%%%%%%%%%%%%%%%%%%%%%%%%%%%%%%%%%%%%

\begin{frame}
\frametitle{Data}

\begin{itemize}

\item I will show you a random sample from the population, but you pay \$200 for each M\&M, and you must buy in \$1000 increments.  

\item That is, you may buy 5, 10, 15, or 20 M\&Ms.

\end{itemize}

\end{frame}

%%%%%%%%%%%%%%%%%%%%%%%%%%%%%%%%%%%%

\subsection{Frequentist inference}

%%%%%%%%%%%%%%%%%%%%%%%%%%%%%%%%%%%%

\begin{frame}
\frametitle{Frequentist inference}

\begin{itemize}

\item Hypotheses:
\begin{itemize}
\item $H_0$: 10\% yellow M\&Ms
\item $H_A$: more than 10\% yellow M\&Ms 
\end{itemize}

\item Your test statistic is the number of yellow M\&Ms you observe in the sample. 

\item The p-value will be the probability of observing this many or more yellow M\&Ms given the null hypothesis is true.

\end{itemize}

\end{frame}

%%%%%%%%%%%%%%%%%%%%%%%%%%%%%%%%%%%%

\begin{frame}
\frametitle{}

\app{Set up -- data}{How many M\&Ms would you buy? Decide as a team and vote.}

\begin{multicols}{4}
\begin{enumerate}[(a)]
\item 5
\item 10
\item 15
\item 20
\end{enumerate}
\end{multicols}

\app{Set up -- significance level}{Then, discuss at what significance level you will reject the null hypothesis.}

\end{frame}

%%%%%%%%%%%%%%%%%%%%%%%%%%%%%%%%%%%%

\begin{frame}
\frametitle{}

Now we will take a sequence of M\&Ms, and you record the number of yellows in the first n draws.

\[ \red{R}\green{G}\yellow{Y}\blue{B}\orange{O} \qquad \blue{B}\blue{B}\green{G}\orange{O}\yellow{Y} \qquad  \yellow{Y}\red{R}\blue{B}\red{R}\red{R} \qquad  \green{G}\orange{O}\red{R}\blue{B}\yellow{Y} \]

\app{Frequentist inference}{
\begin{itemize} 

\item Number of yellows in the first n draws = \rule{1cm}{0.5pt} = k

\item Calculate the p-value using the Binomial distribution: \\

p-value = P(k or more yellows $|$ n, \%yellow is 10\%) = \rule{1cm}{0.5pt}

\item Do you reject the null hypothesis? \rule{1cm}{0.5pt} 

\end{itemize}
}

\vfill

\begin{center}
{\footnotesize See next slide for hints...}
\end{center}

\end{frame}

%%%%%%%%%%%%%%%%%%%%%%%%%%%%%%%%%%%%

\begin{frame}[fragile]
\frametitle{}

Remember $Binomial(n, k) = {n \choose k} p^k (1-p)^{(n-k)}$ \\

$\:$ \\
\pause 

Say you picked $n = 5$, and hence $k = 1$.

\begin{eqnarray*}
&&P(1~or~more~yellows~|~n = 5,~\%yellow~is~10\%) \\
&=& P(K = 1) + P(K = 2) + P(K = 3) + P(K = 4) + P(K = 5) \\
\pause
&=& \left[ {5 \choose 1} 0.1^1 \times 0.9^4 \right]+ \left[{5 \choose 2} 0.1^2 \times 0.9^3 \right] + \cdots + \left[{5 \choose 5} 0.1^5 \times 0.9^0 \right] \\
\pause
&=& 0.32805 + 0.0729 + 0.0081 +  0.00045 + 0.00001 \\
\pause
&\approx& 0.41
\end{eqnarray*}


\end{frame}

%%%%%%%%%%%%%%%%%%%%%%%%%%%%%%%%%%%%

\begin{frame}[fragile]
\frametitle{}

Alternatively: \\
\pause
{\footnotesize
\begin{Verbatim}[frame=single, formatcom=\color{blue}]
# P(K = 1, n = 5, p = 0.1) 
dbinom(1, 5, 0.1)
\end{Verbatim}
}
{\footnotesize
\begin{Verbatim}[frame=single, formatcom=\color{gray}]
0.32805
\end{Verbatim}
}

... and

\pause
{\footnotesize
\begin{Verbatim}[frame=single, formatcom=\color{blue}]
# P(K >= 1, n = 5, p = 0.1) 
sum(dbinom(1:5, 5, 0.1))
\end{Verbatim}
}
{\footnotesize
\begin{Verbatim}[frame=single, formatcom=\color{gray}]
0.40951
\end{Verbatim}
}

%> round(sum(dbinom(1:5,5,0.1)), 2)
%[1] 0.41
%> round(sum(dbinom(2:10,10,0.1)), 2)
%[1] 0.26
%> round(sum(dbinom(3:15,15,0.1)), 2)
%[1] 0.18
%> round(sum(dbinom(4:20,20,0.1)), 2)
%[1] 0.13

\end{frame}

%%%%%%%%%%%%%%%%%%%%%%%%%%%%%%%%%%%%

\subsection{Bayesian inference}

%%%%%%%%%%%%%%%%%%%%%%%%%%%%%%%%%%%%

\begin{frame}
\frametitle{Bayesian inference}

Now we will start over.  Start with 1:1 odds that the percentage of yellows is 10\%:20\%.

\begin{itemize}
\item $H_1$: 10\% yellow M\&Ms $\rightarrow$ P(10\% yellow) = 0.5
\item $H_2$: 20\% yellow M\&Ms $\rightarrow$ P(20\% yellow) = 0.5
\end{itemize}

\pause

$\:$ \\

\app{Bayesian inference}{Using the same data and Bayes' theorem to calculate the probability the percentage of yellow is 10\% and 20\% given the observed data.}

\vfill

\begin{center}
{\footnotesize See next slide for hints...}
\end{center}

\end{frame}

%%%%%%%%%%%%%%%%%%%%%%%%%%%%%%%%%%%%

\begin{frame}
\frametitle{}

P(10\% yellow $|$ data):
{\scriptsize
\begin{eqnarray*}
P(10\% yellow | data) &=& \frac{P(data | 10\% yellow) \times P(10\% yellow)}{P(data)} \\
\pause
&=& \frac{P(data | 10\% yellow) \times P(10\% yellow)}{P(data | 10\% yellow) \times P(10\% yellow) + P(data | 20\% yellow) \times P(20\% yellow)} \\
\pause
&=& \frac{0.33 \times 0.5}{0.33 \times 0.5 + 0.41 \times 0.5} \\
\pause
&=& 0.44
\end{eqnarray*}
}

\end{frame}

%%%%%%%%%%%%%%%%%%%%%%%%%%%%%%%%%%%%

\subsection{Comparison}

%%%%%%%%%%%%%%%%%%%%%%%%%%%%%%%%%%%%

\begin{frame}
\frametitle{Results}

\app{Bayesian vs. Frequentist inference}{
Regardless of the choices you made earlier about $n$, fill out the table below for all possible choices of $n$ and the resulting $k$.
}

\begin{adjustwidth}{-1in}{-1in}

\begin{center}
{\tiny
\renewcommand\arraystretch{2}
\begin{tabular}{p{1.8cm} || c | c || c | c }
		& \multicolumn{2}{c ||}{Frequentist: p-value} & \multicolumn{2}{c}{Bayesian: Posterior} \\
\hline
{\tiny Number of yellow M\&Ms in first} & P(K $\ge$ k $|$ 10\% yellow) & Decision & P(10\% yellow $|$ n,k) & P(20\% yellow $|$ n,k) \\
\hline
$n = 5: k = 1$ & 0.41 & Fail to reject $H_0$ & 0.44 \\
\hline
$n = 10: k = 2$  & & & \\
\hline
$n = 15: k = 3$ & & & \\
\hline
$n = 20: k = 4$ & & &
\end{tabular}
}
\end{center}

\end{adjustwidth}

\end{frame}

%%%%%%%%%%%%%%%%%%%%%%%%%%%%%%%%%%%%

\begin{frame}
\frametitle{Recap}

\begin{itemize}

\item We know that the true \% yellow in these data is 20\%:

\[ \red{R}\green{G}\yellow{Y}\blue{B}\orange{O} \qquad \blue{B}\blue{B}\green{G}\orange{O}\yellow{Y} \qquad  \yellow{Y}\red{R}\blue{B}\red{R}\red{R} \qquad  \green{G}\orange{O}\red{R}\blue{B}\yellow{Y} \]

\item However the Frequentist approach (using p-values) would not allow us to reject the null hypothesis of 10\% yellow.

\item On the other hand, the Bayesian approach yields a higher posterior probability for 20\% yellow.

\end{itemize}

\end{frame}

%%%%%%%%%%%%%%%%%%%%%%%%%%%%%%%%%%%%

\end{document}